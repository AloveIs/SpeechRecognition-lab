\documentclass[onecolumn]{article}
\usepackage[utf8]{inputenc}
\usepackage[swedish,english]{babel}
\usepackage{geometry}

% instead of defining author, write your names below in Group member names
\author{}
\date{}
\title{DT2112 Lab2 Report\\Practical exercise in automatic speech recognition}

\begin{document}
\maketitle
\section*{Group member names}
\begin{itemize}
\item Sp1:
\item Sp2:
\item Sp3:
\end{itemize}

\section*{Grammar explanation and graph (four digits):}
Copy the content of your \verb|four_digits.grm| definition and include the graph you obtained from it. What kind of utterances does this grammar allow? How was this grammar obtained by the rules you defined in \verb|four_digits.grm|?
\clearpage
\section*{Grammar explanation and graph (digit loop):}
Copy the content of your \verb|digit_loop.grm| definition and include the graph you obtained from it. What kind of utterances does this grammar allow? How was this grammar obtained by the rules you defined in \verb|digit_loop.grm|?
\vspace{0.5\textheight}

\section*{Words and their phonetic transcirption}
Report the words in your dictionary and the phonetic transcriptions you have defined.
\clearpage

\section*{Feature extraction parameters}
These are defined in the configuration file \verb|config/features.cfg|, where times are given in 100 ns units. Convert these values into the units specified in the third column of the table. Print the contents using the command: \verb|cat config/features.cfg|.

The sampling frequency of the recordings is not specified in \verb|config/features.cfg|. To extract it, look at one of the files you have recorded. You can use the command \verb|file| to get information about the file, or use the file browser. In the second case right click on the file and choose the ``Properties'' option and then the ``Audio'' tab. The recordings are for each group member in the \verb|Sp$n/train_data| and \verb|Sp$n/test_data| directories\\[5mm]

%\renewcommand{\arraystretch}{1.8}
%\begin{tabular}{|l|l|p{0.3\textwidth}|}
\begin{tabular}{|l|l|l|}
\hline
Parameter & Hint & Value \\ \hline
Sampling frequency (kHz): & check recordings & \hfill (kHz) \\
Analysis window (ms):     & WINDOWSIZE (100 ns units) & \hfill (ms) \\
Frame interval (ms):      & TARGETRATE (100 ns units) & \hfill (ms) \\
Pre-emphasis coeff.       & PREEMPHCOEFF              &    \\
Filterbank \# channels    & NUMCHANS                  &    \\
Energy normalization      & ENORMALISE                &    \\
\# cepstrum coefficients  & NUMCEPS                   &    \\
Hamming                   & USEHAMMING                &    \\
\hline
\end{tabular}

\section*{Answer the following questions:}
How many speech samples are contained in one analysis window?\\[7mm]
How much do consequent analysis windows overlap?\\[7mm]
In a typical four digit utterance that you have recorded, how many analysis windows are used?\\[1mm]

\section*{Acoustic model parameters}
Check the prototype model definition in the file \verb|proto.mmf| and answer the following questions with the help of the HTK Book:\\
What kind of features are used? (hint: defined in the \verb|~o| macro)\\[7mm]
What is the size of the feature vector?\\[7mm]
How many states are used per phoneme?\\[7mm]
Draw the topology (states and transitions) of the prototype model (Hint: \verb|TransP| is the transition probability matrix).
% \begin{tabular}{|p{0.6\textwidth}|c|p{0.15\textwidth}|}
% \hline
% Parameter & Hint & Value \\ \hline
% Context-dependent models (yes/no) & - & No \\
% Tying (yes/no)                    & - & No \\
% Acoustic feature type             & \verb|<MFCC-...>| &  \\
% Acoustic vector size              & \verb|VecSize|    &    \\
% \# states per phone model (incl. start and end states)  & \verb|NumStates| & \\
% \# mixture components per state   & -    & 1 \\
% \hline
% \end{tabular}

\clearpage
\section*{Recognition evaluation}
Speaker dependent results: Matching the test speaker against his/her own trained models. Cross-speaker results: Matching the test speaker against another speaker’s models.\\[5mm]
%\begin{tabular}{|p{0.2\textwidth}|c|c|r|r|r|}
\begin{tabular}{|l|c|c|r|r|r|}
\hline
\textbf{4-digits}  & Training speaker(s) & Test speaker(s) & Accuracy \% & \#Ins & \#Del \\
\hline
Speaker dependent & Sp1 & Sp1 & & & \\\hline
Speaker dependent & Sp2 & Sp2 & & & \\\hline
Speaker dependent & Sp3 & Sp3 & & & \\\hline
Cross-speaker & Sp1 & Sp2 & & & \\\hline
Cross-speaker & Sp1 & Sp3 & & & \\\hline
Cross-speaker & Sp2 & Sp1 & & & \\\hline
Cross-speaker & Sp2 & Sp3 & & & \\\hline
Cross-speaker & Sp3 & Sp1 & & & \\\hline
Cross-speaker & Sp3 & Sp2 & & & \\\hline
\end{tabular}\\[5mm]
%\begin{tabular}{|p{0.2\textwidth}|c|c|r|r|r|}
\begin{tabular}{|l|c|c|r|r|r|}
\hline
\textbf{digit-loop}  & Training speaker(s) & Test speaker(s) & Accuracy \% & \#Ins & \#Del \\
\hline
Speaker dependent & Sp1 & Sp1 & & & \\\hline
Speaker dependent & Sp2 & Sp2 & & & \\\hline
Speaker dependent & Sp3 & Sp3 & & & \\\hline
Cross-speaker & Sp1 & Sp2 & & & \\\hline
Cross-speaker & Sp1 & Sp3 & & & \\\hline
Cross-speaker & Sp2 & Sp1 & & & \\\hline
Cross-speaker & Sp2 & Sp3 & & & \\\hline
Cross-speaker & Sp3 & Sp1 & & & \\\hline
Cross-speaker & Sp3 & Sp2 & & & \\\hline
\end{tabular}

\clearpage
\section*{Discussion of the results}
Common digit confusions:
\vspace{0.2\textheight}

\noindent These might have been caused by:
\vspace{0.4\textheight}

\noindent Compare and discuss the difference between the cross-speaker and the speaker-dependent results. 
\vspace{0.4\textheight}

\noindent How do the performance and types of errors differ between recognition of fixed and non-restricted number of digits (digit loop)?
\vspace{0.3\textheight}

\section*{Experience with the live recogniser}
\noindent how does the recogniser behave when you use it live?
\vspace{0.3\textheight}

\noindent what is the effect of varying the speaking rate?
\clearpage

\noindent what is the effect of varying the insertion penalty?
\vspace{0.3\textheight}

\noindent how does the recogniser cope with long pauses between words? Can you change the behaviour using the insertion penalty?
\vspace{0.3\textheight}

\noindent what happens when you include an optional silence ``word'' between each digits in the grammar?
\clearpage

\noindent what words did you add to the dictionary and grammar? How did the recogniser perform with them?

\end{document}